\chapter{BAKGRUND}
\thispagestyle{fancy}

\section{Raytelligence}
Raytelligence\footnote{\url{http://www.raytelligence.com}} grundades med avsikten att utveckla en radar som kan detektera hjärtslag och andning. Applikationsområden för en sådan givare är många, däribland äldreomsorg. Enligt en undersökning utförd av City University London\cite{elderly_cost} beräknas kostnaden för vård av äldre i Storbritannien att stiga från 11 miljarder pund (1996) till 15 miljarder pund (2040) med 2001 års priser. Raytelligence huvudområde är just äldreomsorg, men de vill gärna öppna upp dörrarna för de många applikationer som givaren skulle kunna användas till. 

\section{Radarenheten}
Raytelligence har utvecklat en radarenhet som arbetar i 60GHz \acs{ISM} bandet och kan detektera hjärtslag och andning, men också rörelse och position. Detta gör då att man kan detektera om någon t\,ex har trillat och inte kommit upp igen. Möjligheter som detta kan ge stora fördelar och besparingar inom t\,ex äldreomsorgen, då onödiga resor mitt i natten ut till de äldres hem kan undvikas. Konsekvensen av dessa besök innebär att de äldre riskerar att vakna mitt i natten och får svårt att somna igen. En alternativ lösning till detta idag är att installera kameror\footnote{\url{http://www.svt.se/nyheter/lokalt/blekinge/kamera-ersatter-hemtjanst-pa-natten}} i deras hem för att kunna övervaka dem på distans. En lösning med radar skulle innebära mindre inskränkning i deras personliga integritet.