\chapter{METOD}
\thispagestyle{fancy}
\section{Webbsystem}

\subsection{Symfony}
För att utveckla en webbapplikation där användaren kan administrera sina radarapplikationer och radarenheter finns en rad olika hjälpmedel som accelererar utvecklingen av denna. Symfony\footnote{\url{www.symfony.com} [2016-10-12]} är ett sådant hjälpmedel, ett \acs{WAF} som låter utvecklaren att fokusera på att skriva bra applikationskod istället för att lägga resurser på underliggande detaljer och säkerhetsfrågor. Många av Symfonys komponenter förekommer i andra populära \acs{WAF}:s såsom Drupal och Laravel. Ramverket öppnar också upp för tredjepartskomponenter i form av s.k bundles. Ett exempel på en sådan är Doctrines \ac{ORM}\footnote{\url{www.doctrine-project.org/projects/orm.html} [2016-10-12]} som har använts frekvent i projektet för integration mot databasen.

% \textbf{Kod och mönster}
\subsection*{Kod och mönster}
Symfony använder sig av välkända designmönster, där bland andra \ac{MVC} spelar en viktig roll. Nedan följer ett exempel för en s.k Controller som visar index-sidan i en webbapplikation:
\newpage

\begin{lstlisting}[caption=Exempel på kod i Symfony, label=symfony]
<?php

class ExampleController extends Controller
{
    /**
    * @Route("/index", name="index")
    */
    public function indexAction(Request $request)
    {
        return $this->render(":public:index.html.twig");
    }
}
\end{lstlisting}
Ovan ser vi även användandet av annoteringar för att specificera sökväg till index-sidan på webbservern, om www.example.com pekar mot servern skulle man nå index med www.example.com/index. Annoteringar används flitigt när webbapplikationer i Symfony skapas, ett ytterligare intressant exempel är i användadet av nämnda Doctrine \acs{ORM} för skapandet av en tabell med två (2) kolumner i databasen:

\begin{lstlisting}[caption=Exempel på kod i Symfony, label=symfony]
<?php

/**
* @ORM\Table(name="users")
* @ORM\Entity
*/
class User
{
    /**
     * @ORM\Id
     * @ORM\Column(type="integer")
     * @ORM\GeneratedValue(strategy="AUTO")
     */
    private $id;

    /**
     * @ORM\Column(name="first_name", type="string")
     */
    private $firstName;
    
}
\end{lstlisting}
\subsection*{Säkerhet}
Vid implementation av webbapplikationen har säkerheten ej åsidosatts. Det finns 3 huvudkomponenter som här är viktiga:
\begin{itemize}
\item Symfony
\item Doctrine \acs{ORM}
\item Twig\footnote{\url{www.twig.sensiolabs.org} [2016-10-12]}
\end{itemize}





\newpage
\section{ZeroMQ}
ZeroMQ ($\oldemptyset$MQ)\cite{ZeroMQ-book} är ett meddelandeorienterat bibliotek som underlättar för användaren att programatiskt hantera sockets inom meddelandedistribution mellan system. Det har stöd för att skicka meddelande på flera olika sätt som in-process, inter-process, \acs{TCP} och multicast.

Följande beskrivning av kommunikationsmönster (Messaging Patterns) återfinns i iMatix, \emph{Code Connected Volume 1}, s. 40\cite{ZeroMQ-book}.\\

De inbyggda kommunikationsmönster i $\oldemptyset$MQ är:
\begin{itemize}
    \item{\textbf{Request-Reply,} som kopplar samman ett set av clienter till ett set av tjänster. Detta är en}
    \item{\textbf{Publish-Subscribe}, som kopplar samman en samling av utgivare till en samling av prenumeranter. Detta är ett datadistributionsmönster.}
    \item{\textbf{Pipeline}, som kopplar samman noder i ett (eng. fan-in/fan-out pattern) som kan ha flera steg och slingor.}
    \item{\textbf{Exclusive pair}, som som enbart kopplar samman två sockets. Detta är ett mönster för att koppla samman två trådar i en process, inte att förväxla med ''normala'' par av sockets.}

\end{itemize}
\newpage
Nedan följer exempel på kod för meddelande mellan server och klient i ZeroMQ skriven i C.

\begin{lstlisting}[caption=Exempel på serverkod i $\oldemptyset$MQ, label=hw-server]
int main (void)
{
    //  Socket to talk to clients
    void *context = zmq_ctx_new ();
    void *responder = zmq_socket (context, ZMQ_REP);
    int rc = zmq_bind (responder, "tcp://*:5555");
    assert (rc == 0);

    while (1) {
        char buffer [10];
        zmq_recv (responder, buffer, 10, 0);
        printf ("Received Hello\n");
        sleep (1);          
        //  Do some 'work'
        zmq_send (responder, "World", 5, 0);
    }
    return 0;
}
\end{lstlisting}

\begin{lstlisting}[caption=Exempel på klientkod i $\oldemptyset$MQ, label=hw-server]
int main (void)
{
    printf ("Connecting to hello world server…\n");
    void *context = zmq_ctx_new ();
    void *requester = zmq_socket (context, ZMQ_REQ);
    zmq_connect (requester, "tcp://localhost:5555");

    int request_nbr;
    for (request_nbr = 0; request_nbr != 10; request_nbr++) {
        char buffer [10];
        printf ("Sending Hello %d…\n", request_nbr);
        zmq_send (requester, "Hello", 5, 0);
        zmq_recv (requester, buffer, 10, 0);
        printf ("Received World %d\n", request_nbr);
    }
    zmq_close (requester);
    zmq_ctx_destroy (context);
    return 0;
}
\end{lstlisting}