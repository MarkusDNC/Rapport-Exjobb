% ****** Preemble ******
\documentclass[12pt,a4paper,twoside,openright]{report}
\usepackage[utf8]{inputenc}
\usepackage[swedish]{babel}

\usepackage[a4paper,bindingoffset=1.5cm,margin=3cm,includeheadfoot,]{geometry}

% För bilder
\usepackage{graphicx}
\graphicspath{{images/}}
\usepackage{wrapfig}
\usepackage[font=small]{caption} % Storlek på caption

% För php kod
\usepackage{listings,xcolor}
\usepackage{inconsolata}

\definecolor{dkgreen}{rgb}{0,.6,0}
\definecolor{dkblue}{rgb}{0,0,.6}
\definecolor{dkyellow}{cmyk}{0,0,.8,.3}

\lstset{
  language        = php,
  basicstyle      = \footnotesize\ttfamily,
  keywordstyle    = \color{dkblue},
  stringstyle     = \color{red},
  identifierstyle = \color{dkgreen},
  commentstyle    = \color{gray},
  emph            =[1]{php},
  emphstyle       =[1]\color{black},
  emph            =[2]{if,and,or,else},
  emphstyle       =[2]\color{dkyellow}}
  
\renewcommand{\lstlistingname}{Kod}

% Font
\usepackage[sc]{mathpazo}
\linespread{1.05} % Palladio needs more leading (space between lines)
\usepackage[T1]{fontenc}

\usepackage{amssymb}
\let\oldemptyset\emptyset
\let\emptyset\varnothing

% Design av kapitelrubrik och numrering
% \usepackage[hcentering,textwidth=150mm,textheight=225mm]{geometry}
\usepackage{fancyhdr}
\usepackage[Bjornstrup]{fncychap}
\ChTitleVar{\large\ChTitleUpperCase}

% Blockquotes
\usepackage[autostyle]{csquotes}  

\usepackage{titlesec}

\titleformat{\section}
{\normalfont\normalsize\uppercase}{\thesection}{1em}{}
\titleformat{\subsection}
{\itshape}{\thesubsection}{1em}{}
\titleformat{\subsubsection}
{\normalfont\small}{\thesubsubsection}{1em}{}

% Definerar färger
\usepackage{xcolor}
\definecolor{acblue}{RGB}{85,123,206}
\definecolor{urlred}{RGB}{130,0,0}
\definecolor{citegreen}{RGB}{0,130,0}

% Akronymer
\usepackage[printonlyused]{acronym}
% Använd argumentet <withpage> för att skriva ut nummer på sida som det första gången används på.


% Prickad linje för Sections i innehållsförteckningen
\usepackage{tocloft}

%För placering av figurer
\usepackage{float}

%För justering av marginaler
% \usepackage{geometry}

%För visa kod i text
\usepackage{listings}

% För bilder
\usepackage{graphicx}
\graphicspath{{images/}}
\usepackage{wrapfig}
\usepackage[font=small]{caption} % Storlek på caption

% Används för url och radbrytning i långa url
\usepackage[hyphens]{url} 
\usepackage[hidelinks,colorlinks=true,linktocpage=true]{hyperref} % Länkar i TOC
\hypersetup{
    colorlinks=true,
    linkcolor=acblue,
    filecolor=magenta,      
    urlcolor=urlred,
    citecolor=citegreen,
}
\usepackage[hyphenbreaks]{breakurl}

 % För flowcharts
\usepackage{tikz}
\usetikzlibrary{shapes.geometric, arrows, calc, shapes, arrows, shapes.multipart}

% Liten font och centrerad text i caption 
\usepackage[font=small, justification=centering]{caption}

% Länkar för TOC
\usepackage[hidelinks]{hyperref}

% Sidhuvud och sidfot
\usepackage{fancyhdr}
\pagestyle{fancy}
\setlength{\headheight}{33pt}
\setlength{\columnsep}{1cm}
\fancyhead{}
\fancyhead[RO]{\date{April 2016}}
\renewcommand{\headrulewidth}{0pt}
\renewcommand{\footrulewidth}{0.25pt}
\fancyfoot[CO]{\thepage}

%Korrekt avstavning
\usepackage[T1]{fontenc}

%Matematiska formler
\usepackage{amsmath}




% ******* Start på dokumentet ******

\begin{document}\sloppy

\begin{titlepage}
    \centering
    \vfill
    {\Huge{
        Examensarbete}\\
        \vspace{1cm}
        \Large{Raytelligent Cloud\\}
        \vskip1cm
        Markus Axelsson\\
        Oskar Lundgren\\
        \vspace{5mm}
        December 2016
    }    
    \vfill
    % Includera grafik
    \vfill

    \begin{center}
        Handledare: Urban Bilstrup\\
        Akademin för informationsteknologi\\
        Halmstad Högskola
    \end{center}
\end{titlepage}
\thispagestyle{empty}

\newpage\phantom{}
\thispagestyle{empty}

\newpage\phantom{}
\thispagestyle{empty}

\newpage\phantom{}
\thispagestyle{empty}

\clearpage
\onecolumn
\pagenumbering{roman}
\section*{Förord}
Denna rapport är resultatet av ett examensarbete på 15 högskolepoäng vilket utförs på halvfart under höstterminen 2016 för Raytelligence ABs räkning. Examensarbetet ingår som ett avslutande moment på vår utbildning till dataingenjörer på Högskolan i Halmstad.\\
\newline
Vi vill tacka Pelle på Raytelligence och Swedish Adrenaline för möjligheten och hans frikostiga stöd i utvecklandet av detta projekt.\\
\newline
Vi vill även tacka vår handledare Urban Bilstrup för hans stöd och vägledning vilket har varit mycket uppskattat under den tid som projektet har fortlöpt.\\
\newline
Markus Axelsson och Oskar Lundgren\\
Halmstad 2016

\newpage
\newpage\null\newpage
%\setcounter{page}{5}
\section*{Sammanfattning}
\subsection*{Abstract}
TODO

\subsection*{Sammanfattning}
TODO

\newpage\phantom{}

\newpage
%\renewcommand{\cftsecleader}{\cftdotfill{\cftdotsep}} % Prickad linje
\newgeometry{top=3cm, bottom=3cm}
\tableofcontents
\thispagestyle{fancy}

\newpage\phantom{}

\restoregeometry

\section*{Ordförklaringar}
\addcontentsline{toc}{section}{Ordförklaringar}
\begin{description}
    \item[End point]{- Den programmerade kopplingen som radarenheten har för stunden. Värdet vid leverans är addressen till Raytelligence molntjänst.}
    \item[Gränssnitt]{- }
    \item[Slutanvändare]{- Forskare, ingenjör eller utvecklare med kunskaper inom programmering och signalbehandling}
    \item[Virtual Machine]{- todo}
    
\end{description}

\newpage
\chapter*{Akrynomer}
\addcontentsline{toc}{section}{Akronymer}
\begin{acronym}
    \acro{API}{Application Programming Interface}  
    \acro{CSRF}{Cross Site Request Forgery}
    \acro{IaaS}{Infrastructure as a Service}
    \acro{IoT}{Internet of Things}
    \acro{ISM}{Industrial, Scientific and Medical}
    \acro{MVC}{Model View controller}
    \acro{ORM}{Object Relational Mapper}
    \acro{PaaS}{Plattform as a Service}
    \acro{SaaS}{Software as a Service}
    \acro{SDK}{Software Development Kit}
    \acro{TCP}{Transmission Control Protocol}
    \acro{WAF}{Web Application Framework}
    \acro{XSS}{Cross Site Scripting}
\end{acronym}



\newpage\phantom{}

\setcounter{page}{1}
\pagenumbering{arabic}

%Inkluderar de olika kapitlen
\chapter{INLEDNING}
\thispagestyle{fancy}
Samhället går mot att fler och fler enheter blir uppkopplade mot internet, vilket ger dem förmågan att kommunicera med varandra och kraftfullare beräkningssystem. Detta öppnar upp för stora möjligheter inom branscher så som sjukvård, äldrevård och hemautomation. Enligt en prognos utförd av IDC beräknas det att till 2020 kommer marknaden för IoT vara värd 1\,700 miljarder dollar. 2021 beräknar telekommunikationsföretaget Ericsson att det kommer finnas 28 miljarder uppkopplade enheter\footnote{\url{http://spectrum.ieee.org/tech-talk/telecom/internet/popular-internet-of-things-forecast-of-50-billion-devices-by-2020-is-outdated}}. Beroende på källa kan denna siffra variera kraftigt särskilt då vissa inte innefattar mobiltelefoner och datorer. I en tidigare uppskattning av Ericsson räknade de med 50 miljarder uppkopplade enheter 2020, en siffra de nu har sänkt.

Raytelligence vill vara del i denna utveckling och koppla upp sin radargivare för att möjliggöra bredare innovation inom radarområdet. Detta projekt kommer därför utvärdera möjligheten att skapa en tjänst som underlättar för ingenjörer och utvecklare att skapa applikationer som utnyttjar radargivaren via en molntjänst.

\section{Syfte och mål}
Syftet med projektet är att skapa ett system för enklare utveckling av radarapplikationer. Detta system skall möjliggöra kommunikation mellan radargivare, molntjänst och mobilapplikation. Systemet kommer utgöra en bas för utvecklare att bygga på, vilken ska låta ingenjörer, forskare och utvecklare implementera egna molnbaserade lösningar med radargivaren. 

Målet är skapa ett system mellan en eller flera radarsensorer, en molntjänst samt en mobilapplikation. Användaren ska kunna registrera sig själv och sin(a) radargivare på en hemsida, och sedan implementera egna applikationer med sin(a) radargivare. Radargivaren är tänkt att kunna utföra en viss del av beräkningarna på sitt eget kort men även kunna avlastas genom att utföra större beräkningar i en molntjänst.

\section{Frågeställning}
\vspace{3mm}

\subsection*{Bakgrund och teori}
\begin{itemize}
    \item Vilka produkter och tjänster finns på marknaden idag som motsvarar detta projekt?
\end{itemize}

\subsection*{Utbildningsfas}

\begin{itemize}
    \item Vilket/vilka programmeringsspråk skall användas för att lösa uppgiften?
    \item Vilka kommunikationsmetoder behöver användas för att erbjuda ett robust system?
    \item Hur binder slutanvändaren en specifik givare till en specifik applikation?
    \item Vad behöver systemets publika \ac{API} innehålla för att kunna skapa den grundläggande applikation som avses ingå  projektet? 
\end{itemize}

\section{Avgränsningar}
Projekt av denna karaktär väcker ofta frågor om personlig integritet och säkerhet. Projektet kommer att förbereda för en säker miljö, men ingen extra implementation av säkerhetsåtgärder kommer att ske. 

På grund av projektets omfattning  kommer bibliotek för signalbehandling endast att inkluderas om tid finns samt endast ett begränsat \ac{API} kommer att erbjudas. Radarapplikationen som utvecklas i projektet kommer därför att vara en grundläggande applikation för att bevisa funktionaliteten i systemet. 

Endast ett programmeringsspråk kommer att stödjas för användarens applikationer där koden ligger i en (1) fil. 

I ett inledande skede skall en radarenhet endast kunna användas i en applikation. Inte heller optimering av skalning för molntjänst kommer att ske då detta kräver flera radarenheter och applikationer för testning och optimering.

I en framtida version är det önskvärt med ett \ac{SDK} för att underlätta utveckling av en applikation. Detta kommer inte att utvecklas inom ramen för detta projekt.
\chapter{BAKGRUND}
\thispagestyle{fancy}

\section{Raytelligence}
Raytelligence\footnote{\url{http://www.raytelligence.com}} grundades med avsikten att utveckla en radar som kan detektera hjärtslag och andning. Applikationsområden för en sådan givare är många, däribland äldreomsorg. Enligt en undersökning utförd av City University London\cite{elderly_cost} beräknas kostnaden för vård av äldre i Storbritannien att stiga från 11 miljarder pund (1996) till 15 miljarder pund (2040) med 2001 års priser. Raytelligence huvudområde är just äldreomsorg, men de vill gärna öppna upp dörrarna för de många applikationer som givaren skulle kunna användas till. 

\section{Radarenheten}
Raytelligence har utvecklat en radarenhet som arbetar i 60GHz \acs{ISM} bandet och kan detektera hjärtslag och andning, men också rörelse och position. Detta gör då att man kan detektera om någon t\,ex har trillat och inte kommit upp igen. Möjligheter som detta kan ge stora fördelar och besparingar inom t\,ex äldreomsorgen, då onödiga resor mitt i natten ut till de äldres hem kan undvikas. Konsekvensen av dessa besök innebär att de äldre riskerar att vakna mitt i natten och får svårt att somna igen. En alternativ lösning till detta idag är att installera kameror\footnote{\url{http://www.svt.se/nyheter/lokalt/blekinge/kamera-ersatter-hemtjanst-pa-natten}} i deras hem för att kunna övervaka dem på distans. En lösning med radar skulle innebära mindre inskränkning i deras personliga integritet.
\chapter{TEORI}
\thispagestyle{fancy}

\section{Internet Of Things}
\ac{IoT} handlar om hur saker, bland annat i vår vardag, blir uppkopplade till Internet. I dag finns det en uppsjö av uppkopplade enheter för hemmet t\,ex lampor, hemövervakning, kylskåp och lås. Men inte bara våra hem blir uppkopplade utan även våra bilar. Exempel på detta är biltillverkaren Tesla som skickar ut uppdateringar i mjukvaran till deras kunders bilar och därmed kan erbjuda förbättringar och nya funktioner utan att kunden behöver göra något\footnote{\url{https://www.wired.com/insights/2014/02/teslas-air-fix-best-example-yet-internet-things/}}.

Innebörden är helt enkelt att sakerna i vår omgivning blir mer och mer uppkopplade till ett nätverk och kan interagera med användaren och andra noder. Mitt i allt detta finns nästan alltid en molntjänst (Cloud computing) som hanterar den massiva mängd data som genereras av alla dessa noder.

\section{Distribuerade system}
\begin{displayquote} \emph{Ett distribuerat system är en samling av oberoende datorer som visas för sina användare som ett enda sammanhängande system}.\\(Tanenbaum and Steen, 2007, s. 2)\cite{Tanenbaum}

\end{displayquote}

\section{Cloud computing}
Cloud computing, på svenska datormoln, är ett begrepp som innefattar datorberäkningar som erbjuds över Internet. Detta kan vara lagring av data eller tunga databeräkningar som ofta sköts av kraftiga servers som befinner sig på en helt annan plats än användaren. Exempel på lagringstjänster kan vara välkända tjänster så som Dropbox, Google Drive och Apple iCloud. För mer skräddasydda tjänster finns t\,ex Amazon S3.

\subsection{Software as a Service}
\ac{SaaS}\cite{cloud_computing} är en tjänst som erbjuder användaren att köra applikationer som tillhandahålls åt användaren på en molntjänst. Användaren kan då använda applikationen via en tunn klient så som en webbläsare eller lättare programvara. Exempel på detta kan vara webbmail eller kontorsprogram så som Google Docs. Det enda som då krävs av användaren är t\,ex en webbläsare då exekvering och lagring av applikationen sker i molntjänsten och inte lokalt på användarens dator bortsett från begränsade användarspecifika inställningar och funktioner.

\subsection{Platform as a Service}
\ac{PaaS}\cite{cloud_computing} är en tjänst som erbjuder användaren att på en plattform applicera och exekvera sin egen eller någon annans programvara i en molntjänst. Dock styr inte användaren något av det underliggande lagret så som operativsystem, lagring och nätverk utan bara konfigurationen av sin egen applikation. 

\subsection{Infrastructure as a Service}
\ac{IaaS}\cite{cloud_computing} är en tjänst som erbjuder användaren total kontroll över sin instans av molntjänsten. Leverantören av molntjänsten tillhandahåller processor, minne, lagring och övriga kritiska komponenter och tjänster nödvändiga för att användaren skall kunna installera programvara, både i form av operativsystem och applikationer. Användaren har inte kontroll över underliggande molnstruktur men styr över operativsystem, lagring, applikationer och konfigurationer så som brandvägg och portar.

\section{Webbsystem}
\subsection{Säkerhet}
Vid utveckling av publika webbsystem finns en rad problem och säkerhetsfrågor som man bör vara medveten om. Det handlar både om att skydda systemets data, men även användarens integritet och dennes data. Systemets data kan vara känsliga uppgifter i en databas såsom medicinska journaler och personuppgifter. 

SQL Injections är ett vanligt sätt för en illasinnad användare att utnyttja för att komma över data från en databas kopplad till exempelvis en hemsida. Detta kan ha förödande konsekvenser för den personliga integriteten men även ett företags affärsmodell och trovärdighet kan komma att frågasättas.


\ac{XSS} är en teknik för angripare att köra skadlig kod i offrets webbläsare. Detta kan ske om en användares input renderas direkt i webbläsaren. Den skadliga koden kan då ha tillgång till samma objekt som resten av hemsidan, såsom session cookies, med vilka angriparen kan utge sig för att vara offret.

\ac{CSRF} är en metod för en tredjeparts hemsida att sända förfrågningar till målsidan (exempelvis en bank) genom att använda målsidans cookies och session. Detta kan ske om användaren exempelvis är inloggad på sin bank i en flik i webbläsaren och med en annan flik besöker en illvillig hemsida.
\subsection{Ramverk för webbutveckling}


\ac{WAF} är ramverk som ofta är baserade på \ac{MVC}, skapade för att underlätta utvecklandet av webbsystem\cite{waf} och webbapplikationer. Avsikten med dessa är att utvecklaren inte skall behöva fokusera på vanligt förekommande implementationer av funktioner och säkerhet. På så sätt behöver användaren av ramverket inte lägga tid på att t\,ex skydda sin databas mot SQL injections och \ac{XSS} utan detta sköts istället av ramverket.

\chapter{METOD}
\thispagestyle{fancy}
\section{Webbsystem}

\subsection{Symfony}
För att utveckla en webbapplikation där användaren kan administrera sina radarapplikationer och radarenheter finns en rad olika hjälpmedel som accelererar utvecklingen av denna. Symfony\footnote{\url{www.symfony.com} [2016-10-12]} är ett sådant hjälpmedel, ett \acs{WAF} som låter utvecklaren att fokusera på att skriva bra applikationskod istället för att lägga resurser på underliggande detaljer och säkerhetsfrågor. Många av Symfonys komponenter förekommer i andra populära \acs{WAF}:s såsom Drupal och Laravel. Ramverket öppnar också upp för tredjepartskomponenter i form av s.k bundles. Ett exempel på en sådan är Doctrines \ac{ORM}\footnote{\url{www.doctrine-project.org/projects/orm.html} [2016-10-12]} som har använts frekvent i projektet för integration mot databasen.

% \textbf{Kod och mönster}
\subsection*{Kod och mönster}
Symfony använder sig av välkända designmönster, där bland andra \ac{MVC} spelar en viktig roll. Nedan följer ett exempel för en s.k Controller som visar index-sidan i en webbapplikation:
\newpage

\begin{lstlisting}[caption=Exempel på kod i Symfony, label=symfony]
<?php

class ExampleController extends Controller
{
    /**
    * @Route("/index", name="index")
    */
    public function indexAction(Request $request)
    {
        return $this->render(":public:index.html.twig");
    }
}
\end{lstlisting}
Ovan ser vi även användandet av annoteringar för att specificera sökväg till index-sidan på webbservern, om www.example.com pekar mot servern skulle man nå index med www.example.com/index. Annoteringar används flitigt när webbapplikationer i Symfony skapas, ett ytterligare intressant exempel är i användadet av nämnda Doctrine \acs{ORM} för skapandet av en tabell med två (2) kolumner i databasen:

\begin{lstlisting}[caption=Exempel på kod i Symfony, label=symfony]
<?php

/**
* @ORM\Table(name="users")
* @ORM\Entity
*/
class User
{
    /**
     * @ORM\Id
     * @ORM\Column(type="integer")
     * @ORM\GeneratedValue(strategy="AUTO")
     */
    private $id;

    /**
     * @ORM\Column(name="first_name", type="string")
     */
    private $firstName;
    
}
\end{lstlisting}
\subsection*{Säkerhet}
Vid implementation av webbapplikationen har säkerheten ej åsidosatts. Det finns 3 huvudkomponenter som här är viktiga:
\begin{itemize}
\item Symfony
\item Doctrine \acs{ORM}
\item Twig\footnote{\url{www.twig.sensiolabs.org} [2016-10-12]}
\end{itemize}





\newpage
\section{ZeroMQ}
ZeroMQ ($\oldemptyset$MQ)\cite{ZeroMQ-book} är ett meddelandeorienterat bibliotek som underlättar för användaren att programatiskt hantera sockets inom meddelandedistribution mellan system. Det har stöd för att skicka meddelande på flera olika sätt som in-process, inter-process, \acs{TCP} och multicast.

Följande beskrivning av kommunikationsmönster (Messaging Patterns) återfinns i iMatix, \emph{Code Connected Volume 1}, s. 40\cite{ZeroMQ-book}.\\

De inbyggda kommunikationsmönster i $\oldemptyset$MQ är:
\begin{itemize}
    \item{\textbf{Request-Reply,} som kopplar samman ett set av clienter till ett set av tjänster. Detta är en}
    \item{\textbf{Publish-Subscribe}, som kopplar samman en samling av utgivare till en samling av prenumeranter. Detta är ett datadistributionsmönster.}
    \item{\textbf{Pipeline}, som kopplar samman noder i ett (eng. fan-in/fan-out pattern) som kan ha flera steg och slingor.}
    \item{\textbf{Exclusive pair}, som som enbart kopplar samman två sockets. Detta är ett mönster för att koppla samman två trådar i en process, inte att förväxla med ''normala'' par av sockets.}

\end{itemize}
\newpage
Nedan följer exempel på kod för meddelande mellan server och klient i ZeroMQ skriven i C.

\begin{lstlisting}[caption=Exempel på serverkod i $\oldemptyset$MQ, label=hw-server]
int main (void)
{
    //  Socket to talk to clients
    void *context = zmq_ctx_new ();
    void *responder = zmq_socket (context, ZMQ_REP);
    int rc = zmq_bind (responder, "tcp://*:5555");
    assert (rc == 0);

    while (1) {
        char buffer [10];
        zmq_recv (responder, buffer, 10, 0);
        printf ("Received Hello\n");
        sleep (1);          
        //  Do some 'work'
        zmq_send (responder, "World", 5, 0);
    }
    return 0;
}
\end{lstlisting}

\begin{lstlisting}[caption=Exempel på klientkod i $\oldemptyset$MQ, label=hw-server]
int main (void)
{
    printf ("Connecting to hello world server…\n");
    void *context = zmq_ctx_new ();
    void *requester = zmq_socket (context, ZMQ_REQ);
    zmq_connect (requester, "tcp://localhost:5555");

    int request_nbr;
    for (request_nbr = 0; request_nbr != 10; request_nbr++) {
        char buffer [10];
        printf ("Sending Hello %d…\n", request_nbr);
        zmq_send (requester, "Hello", 5, 0);
        zmq_recv (requester, buffer, 10, 0);
        printf ("Received World %d\n", request_nbr);
    }
    zmq_close (requester);
    zmq_ctx_destroy (context);
    return 0;
}
\end{lstlisting}
\newgeometry{left=3cm,right=3cm}
\begin{thebibliography}{1}

\bibitem{cloud_computing}
Mell, P. and Grance, T. (2011). \emph{The NIST Definition of Cloud Computing}. 1st ed. [ebook] pp.2-3. Available at: \url{http://faculty.winthrop.edu/domanm/csci411/Handouts/NIST.pdf} [Accessed 5 Oct. 2016].

\bibitem{waf}
Lu M, Wing-lok Yeung. \emph{A framework for effective commercial Web application development}. Internet Research 1998;8(2):166-173.

\bibitem{symfony_guide}
Zaninotto, F. and Potencier, F. (2007). \emph{The definitive guide to symfony}. Berkeley, Calif.: Apress.

\bibitem{symfony_pdf}
Jarmolowicz, J., Zabierowski, W. and Napieralski, A. (2016). \emph{Presentation of improvements for PHP}. 1st ed. [ebook] p.1. Available at: \url{https://www.researchgate.net/profile/Wojciech_Zabierowski/publication/251883714_Presentation_of_improvements_for_PHP_programmers_based_on_Symfony_framework._Creation_of_example_portal_and_description_of_used_technology/links/00b7d538f112357a92000000.pdf} [Accessed 3 Oct. 2016].

\bibitem{elderly_cost}
Karlsson, M., Mayhew, L., Plumb, R. and Rickayzen, B. (2016). 2006. 1st ed. [ebook] London, UK: City University, p.198. Available at: \url{http://www.actuaries.org/IAAHS/OnlineJournal/2006-2/long\%20term\%20care\%20UK.pdf} [Accessed 5 Oct. 2016].

\bibitem{IoT}
Gubbi, J., Buyya, R., Marusic, S. and Palaniswamia, M. (2016). \emph{Internet of Things (IoT): A Vision, Architectural Elements, and Future Directions}. 1st ed. [ebook] Melbourne, p.1. Available at: \url{https://arxiv.org/pdf/1207.0203.pdf} [Accessed 5 Oct. 2016].

\bibitem{Tanenbaum}
Tanenbaum, A. and Steen, M. (2007). \emph{Distributed systems}. Upper Saddle River, NJ: Pearson Prentice Hall.

\bibitem{ZeroMQ-book}
Hintjens, P. (2013). \emph{Code connected}. [S.l.]: [s.n.].


\end{thebibliography}


\end{document}